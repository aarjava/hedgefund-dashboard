\pdfoutput=1
\documentclass[11pt]{article}
\usepackage[margin=1in]{geometry}
\usepackage{graphicx}
\usepackage{booktabs}
\usepackage{amsmath}
\usepackage{amssymb}
\usepackage{float}
\usepackage[numbers]{natbib}
\usepackage[colorlinks=true,linkcolor=blue,citecolor=blue,urlcolor=blue]{hyperref}

\title{Volatility Regimes and Trend-Following Performance in U.S. Equities: An Empirical Deconstruction}
\author{Aarjav Ametha}
\date{February 2026}

\begin{document}
\maketitle

\begin{abstract}
This paper evaluates a standard trend-following rule ($P_t > SMA_{50,t}$) on SPY over a 33-year sample (1993-01-29 to 2026-02-13). Unconditionally, the strategy underperforms buy-and-hold on Sharpe (\textbf{0.32} vs. \textbf{0.46}) and CAGR (\textbf{3.83\%} vs. \textbf{8.68\%}), but materially improves downside containment (MaxDD \textbf{-36.46\%} vs. \textbf{-56.47\%}). A regime-conditional decomposition reveals structural asymmetry: quality is concentrated in Low Volatility (Sharpe 1.55) and decays through volatility expansion (Normal 0.11, High 0.21). Walk-forward validation remains directionally consistent (OOS Sharpe 0.33 across 61 test windows), supporting robustness of the central result.
\end{abstract}

\section{Introduction}
Trend-following in equities is frequently framed as ``crisis alpha,'' i.e., superior return quality during volatility shocks. We test that claim by conditioning performance on out-of-sample volatility regimes instead of relying on unconditional averages. The framing is consistent with prior evidence on moving-average rules and trend-following across assets \citep{brock1992simple,moskowitz2012time,hurst2017century} as well as broader momentum evidence in equities \citep{jegadeesh1993returns}.

The empirical profile in this sample is not a smile; it is closer to a checkmark. Performance quality is strongest in Low Volatility and degrades in Normal and High Volatility states. The strategy still provides material drawdown truncation, but that benefit is risk-management oriented, not broad crisis-state alpha; practitioner treatments also emphasize this allocation lens \citep{antonacci2014dual}.

\subsection{Hypotheses and contributions}
Hypotheses tested:
\begin{itemize}
\item \textbf{H1 (Crisis alpha):} trend-following quality is highest in High Volatility states.
\item \textbf{H2 (Low-vol dominance):} trend-following quality is highest in Low Volatility states.
\item \textbf{H3 (Transition bleed):} the largest quality decay occurs during $Low \rightarrow Normal$ transitions.
\end{itemize}

Contributions:
\begin{itemize}
\item OOS regime decomposition with expanding-window state labels.
\item Transition-level microstructure diagnostics for where quality is lost.
\item Robustness stack across walk-forward, cost/rebalance, SMA sweep, and cross-asset checks.
\item Explicit hypothesis-to-evidence mapping.
\end{itemize}

\section{Data and Methodology}
\subsection{Sample and signal}
Instrument: SPY (robustness assets: QQQ and IWM). Raw sample window: 1993-01-29 to 2026-02-13. Effective analysis starts 1993-04-12 after indicator warm-up and out-of-sample regime eligibility.

Trading signal:
\begin{equation}
\text{Position}_t =
\begin{cases}
1 & \text{if } P_t > SMA_{50,t} \\
0 & \text{otherwise}
\end{cases}
\end{equation}

Execution assumptions: monthly rebalance, 10 bps turnover cost.

\subsection{Regime definition and validation}
Regimes are defined from annualized 21-day realized volatility using expanding-window quantiles:
Low ($<25$th percentile), Normal ($25$th--$75$th), and High ($>75$th). This prevents look-ahead in threshold construction.

Validation stack:
\begin{itemize}
\item Bootstrap confidence intervals and p-values for strategy-minus-benchmark differences by regime.
\item High-minus-Normal spread test for strategy-only quality decay.
\item Rolling walk-forward evaluation (24-month train / 6-month test).
\end{itemize}

\section{Results}
\subsection{Unconditional performance: insurance cost vs. tail truncation}
\begin{itemize}
\item Strategy CAGR: 3.83\%; Benchmark CAGR: 8.68\%.
\item Strategy Sharpe: 0.32; Benchmark Sharpe: 0.46.
\item Strategy MaxDD: -36.46\%; Benchmark MaxDD: -56.47\%.
\item Strategy Win Rate: 34.81\%; Benchmark Win Rate: 53.66\%.
\end{itemize}

\begin{figure}[H]
\centering
\includegraphics[width=0.92\linewidth]{figures/fig_equity_curves.png}
\caption{Figure 1. SPY equity curves (log scale), trend strategy vs. buy-and-hold.}
\label{fig:equity}
\end{figure}

\begin{figure}[H]
\centering
\includegraphics[width=0.92\linewidth]{figures/fig_drawdowns.png}
\caption{Figure 2. SPY drawdown curves showing tail-risk truncation.}
\label{fig:drawdown}
\end{figure}

Figures~\ref{fig:equity} and \ref{fig:drawdown} show the key trade-off: lower trend participation in long bull runs, but materially reduced tail-depth and faster drawdown recovery dynamics.

\subsection{Regime anomaly: where quality actually lives}
\begin{figure}[H]
\centering
\includegraphics[width=0.75\linewidth]{figures/fig_regime_frequency.png}
\caption{Figure 3. OOS volatility-regime occupancy for SPY.}
\label{fig:regimefreq}
\end{figure}

\begin{figure}[H]
\centering
\includegraphics[width=0.82\linewidth]{figures/fig_conditional_sharpe.png}
\caption{Figure 4. Regime-conditional Sharpe ratios (strategy vs. benchmark).}
\label{fig:condsharpe}
\end{figure}

Figure~\ref{fig:regimefreq} confirms occupancy is meaningful in all states (Low 24.04\%, Normal 45.72\%, High 29.86\%), so the conditional profile is not a sparse-sample artifact.
Figure~\ref{fig:condsharpe} directly shows the checkmark profile.
\begin{itemize}
\item Strategy Sharpe by regime: Low 1.55, Normal 0.11, High 0.21.
\item Benchmark Sharpe by regime: Low 1.74, Normal 0.43, High 0.45.
\item High-minus-Normal strategy Sharpe spread: 0.10 (95\% CI [-0.75, 0.89], p=0.799).
\end{itemize}

\subsection{Transition microstructure and robustness}
\begin{figure}[H]
\centering
\includegraphics[width=0.70\linewidth]{figures/fig_transition_matrix.png}
\caption{Figure 5. Volatility regime transition matrix.}
\label{fig:transition}
\end{figure}

Figure~\ref{fig:transition} provides the transition diagnostics:
\begin{itemize}
\item $P(\text{High}_t \mid \text{High}_{t-1})$ = 96.19\%.
\item $P(\text{Normal}_t \mid \text{Low}_{t-1})$ = 6.79\%.
\item Sharpe during $Low \rightarrow Normal$ transitions = -5.09.
\end{itemize}

The $Low \rightarrow Normal$ handoff is the primary bleed regime: trend smoothness breaks, realized volatility expands, and lagged signals adapt late.

\begin{figure}[H]
\centering
\includegraphics[width=0.84\linewidth]{figures/fig_sma_sweep.png}
\caption{Figure 6. SMA lookback sweep by regime-level Sharpe ratio.}
\label{fig:smasweep}
\end{figure}

\begin{figure}[H]
\centering
\includegraphics[width=0.92\linewidth]{figures/fig_robustness_assets.png}
\caption{Figure 7. Cross-asset robustness across SPY, QQQ, and IWM.}
\label{fig:crossasset}
\end{figure}

Additional robustness diagnostics:
\begin{itemize}
\item OOS walk-forward summary: CAGR 3.61\%, Sharpe 0.33, MaxDD -33.22\%, periods 61.
\item Cost sensitivity: strategy CAGR declines from 4.29\% (0 bps) to 2.01\% (50 bps).
\item Rebalance sensitivity: strategy Sharpe is 0.17 (daily) vs. 0.32 (monthly).
\item Baseline comparison: SMA50 Sharpe 0.33 vs. SMA200 Sharpe 0.70 on common sample.
\end{itemize}

Figures~\ref{fig:smasweep} and \ref{fig:crossasset} align with the core interpretation: low-volatility trend quality is persistent across parameterizations and strongest in momentum-rich indices.

\section{Hypothesis-to-Evidence Alignment}
\begin{itemize}
\item \textbf{H1 (Crisis alpha): Rejected.} Figure~\ref{fig:condsharpe} shows High-vol strategy Sharpe (0.21) below Low-vol strategy Sharpe (1.55) and below High-vol benchmark Sharpe (0.45).
\item \textbf{H2 (Low-vol dominance): Supported.} Figure~\ref{fig:condsharpe} shows highest strategy quality in Low volatility, and Figure~\ref{fig:smasweep} shows this ordering is robust across lookbacks.
\item \textbf{H3 (Transition bleed): Supported.} Figure~\ref{fig:transition} isolates severe quality decay in the $Low \rightarrow Normal$ handoff (Sharpe -5.09).
\end{itemize}

\section{Appendix: Inference and Robustness Tables}
\begin{table}[H]
\centering
\caption{Strategy minus Benchmark by regime (bootstrap).}
\label{tab:inf_regime}
\resizebox{\linewidth}{!}{%
\begin{tabular}{lrrrrrrrr}
\toprule
Regime & Sharpe Diff (S-B) & Sharpe CI Low & Sharpe CI High & Sharpe p-value & CAGR Diff (pp) & CAGR CI Low (pp) & CAGR CI High (pp) & CAGR p-value \\
\midrule
Low & -0.191 & -1.148 & 0.819 & 0.732 & -2.39\% & -11.43\% & 7.16\% & 0.656 \\
Normal & -0.322 & -1.036 & 0.392 & 0.519 & -4.61\% & -14.65\% & 4.66\% & 0.518 \\
High & -0.233 & -1.100 & 0.668 & 0.666 & -6.97\% & -30.53\% & 13.57\% & 0.616 \\
\bottomrule
\end{tabular}
}
\end{table}

\begin{table}[H]
\centering
\caption{High minus Normal differences (strategy only).}
\label{tab:inf_hn}
\resizebox{\linewidth}{!}{%
\begin{tabular}{lrrrr}
\toprule
Metric & Estimate & CI Low & CI High & p-value \\
\midrule
Sharpe (High - Normal) & 0.103 & -0.751 & 0.891 & 0.799 \\
CAGR (High - Normal) & 1.48\% & -9.47\% & 12.41\% & 0.809 \\
\bottomrule
\end{tabular}
}
\end{table}

\begin{table}[H]
\centering
\caption{Transaction cost sensitivity.}
\label{tab:cost}
\resizebox{\linewidth}{!}{%
\begin{tabular}{lrrrrr}
\toprule
Cost (bps) & Strategy CAGR & Strategy Sharpe & Strategy MaxDD & Delta CAGR vs Buy-Hold & Delta Sharpe vs Buy-Hold \\
\midrule
0 & 4.29\% & 0.360 & -33.47\% & -4.39\% & -0.104 \\
5 & 4.06\% & 0.341 & -34.98\% & -4.62\% & -0.124 \\
10 & 3.83\% & 0.322 & -36.46\% & -4.85\% & -0.143 \\
20 & 3.37\% & 0.284 & -39.31\% & -5.31\% & -0.181 \\
50 & 2.01\% & 0.169 & -47.14\% & -6.67\% & -0.296 \\
\bottomrule
\end{tabular}
}
\end{table}

\begin{table}[H]
\centering
\caption{Rebalance frequency sensitivity.}
\label{tab:rebal}
\resizebox{\linewidth}{!}{%
\begin{tabular}{lrrrrr}
\toprule
Rebalance Frequency & Strategy CAGR & Strategy Sharpe & Strategy MaxDD & Delta CAGR vs Buy-Hold & Delta Sharpe vs Buy-Hold \\
\midrule
Daily & 1.88\% & 0.174 & -52.76\% & -6.80\% & -0.291 \\
Weekly & 3.03\% & 0.273 & -40.66\% & -5.66\% & -0.191 \\
Monthly & 3.83\% & 0.322 & -36.46\% & -4.85\% & -0.143 \\
\bottomrule
\end{tabular}
}
\end{table}

\begin{table}[H]
\centering
\caption{Baseline signal comparison.}
\label{tab:baseline}
\resizebox{\linewidth}{!}{%
\begin{tabular}{lrrr}
\toprule
Model & CAGR & Sharpe & Max Drawdown \\
\midrule
BuyHold & 8.68\% & 0.461 & -56.47\% \\
SMA50 & 3.95\% & 0.330 & -36.46\% \\
SMA200 & 8.71\% & 0.696 & -26.29\% \\
\bottomrule
\end{tabular}
}
\end{table}

\section{Interpretation and Limits}
Bootstrap inference supports directionality but indicates non-trivial uncertainty in several effect-size gaps. Strategy-minus-benchmark Sharpe p-values by regime are: Low 0.732, Normal 0.519, High 0.666. Therefore conclusions should be read as structural (state-dependent quality and robust drawdown truncation), not as point-estimate precision claims.

\section{Conclusion}
For U.S. equities in this sample, trend-following is best interpreted as a regime-dependent exposure controller rather than a universal crisis-alpha engine. The strongest improvement path is transition-aware and volatility-adaptive signal speed, especially around the $Low \rightarrow Normal$ state break where performance decay is most severe.

\bibliographystyle{plainnat}
\bibliography{references}
\end{document}
